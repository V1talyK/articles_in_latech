%%
%% ****** Generated 24.03.23 by LJM TeX-constructor******
%%
\documentclass{article}

\usepackage{amssymb}
\usepackage{amsfonts}
\usepackage[tbtags]{amsmath}
\usepackage{amscd}
\usepackage{amsthm}			% Продвинутая математика
\usepackage{mathtext}
\usepackage{cmap}
\usepackage[T2A]{fontenc}
\usepackage[utf8]{inputenc}			% cp1251
\usepackage[english, russian]{babel}
%\usepackage{literat}
\usepackage{pifont}
\usepackage{bm}
\usepackage{array}			% Ширина столбиков в массиве
\usepackage{dcolumn}
\usepackage{hhline}
\usepackage{multirow}
\usepackage{graphicx}
\usepackage{rotating}
\usepackage{calc}
\usepackage{tabularx}
\usepackage{afterpage}
\usepackage{ifthen}
\usepackage{caption2}
\usepackage{substr}
\usepackage[mathscr]{eucal}  %% My addition
\usepackage{mathrsfs}        %% My addition
\usepackage{hypbmsec}        %% My addition
\usepackage{latexsym}        %% My addition
\usepackage{xypic}           %% My addition
\RequirePackage{soul}
\RequirePackage{verbatim}    %% My addition
\RequirePackage{chapterbib}
\RequirePackage{enumerate}


\usepackage[shortcuts]{extdash}
\usepackage{ragged2e}
\usepackage{etoolbox}
\usepackage{lipsum}

%\usepackage{flafter}
\usepackage[section,above,below]{placeins}

\usepackage{indentfirst}
\usepackage[a4paper, top=20mm, left=30mm, right=20mm, bottom=25mm]{geometry}


\setcounter{page}{3}

\begin{document}
\title{Influence of weight coefficients on the solution of the inverse problem} % for running heads
\author{Kosyakov} % for running heads
\author{Legostaev}

%\noaffiliation % If the author does not specify a place of work.

%\firstcollaboration{(Submitted by ) } % Add if you know submitter.
%\lastcollaboration{ }


\begin{abstract} % You shouldn't use formulas and citations in the abstract.
In the oil industry, there is a noticeable tendency to proxy modeling of various levels of complexity to perform operational predictive calculations, in particular, machine learning methods that are actively developing in the context of digitalization and intellectualization of production processes. In this paper, using the example of a synthetic oil reservoir model development element, we present an approach to the joint use of a physically meaningful fluid flow model and machine learning methods for solving adaptation and prediction problems. A feature of the considered synthetic model is the presence of a pronounced zonal inhomogeneity of the permeability field. Within the framework of the proposed approach, a single-phase filtration model, simplified in comparison with the original formulation, was used, which was history-matched by restoring the field of reservoir filtration parameters using a network of radial basis functions. Based on the reconstructed field, the connection coefficients between the wells were calculated, which qualitatively and quantitatively correspond to the true well connection. The next step was to train the recurrent neural network in order to predict the water cut of the produced fluid. The use of a recurrent neural network made it possible to reproduce the characteristic non-monotonic behavior of the water cut of the produced fluid, caused by non-stationary modes of operation of injection and production wells. A combination of the presented models makes it possible to predict the volume of the produced fluid and its phase composition. To assess the predictive properties of the models, the actual data set was divided into training and test intervals.
\end{abstract}

\textbf{Keywords}:flow through porous medium, reservoir mathematical simulation, inverse problem, adjoint problem, machine learning, radial basis functions, recurrent neural networks % Include keywords separeted by comma.

\maketitle

% Text of article starts here.



\section{INTRODUCTION}

В настоящее время в нефтедобывающей отрасли наряду с использованием традиционных гидродинамических моделей широкое применение получило использование упрощённых прокси-моделей. Использование упрощённых моделей позволяет сократить вычислительные затраты, а также снизить требования к качеству и полноте исходных данных. Самыми ресурсоёмкими задачами, как правило, являются обратные и оптимизационные задачи, поэтому развитие и применение прокси-моделирования для подобного рода задач является актуальным. Помимо физически содержательных моделей (материальный баланс \cite{mus1}, супер элементы \cite{maz}, CRM \cite{bek}, линии тока \cite{pot} и т.д.) развиваются подходы основанные на применении методов машинного обучения \cite{tem}, \cite{ill}, \cite{uma}. Например, рекуррентные нейронные сети получили применение в области прогнозирования временных рядов и, в частности, режимов работы скважин \cite{gop}. С учётом растущей цифровизацией промысла такие подходы являются весьма перспективными. Однако, применение подходов, основанных только на методах машинного обучения, в качестве прогнозирующих моделей не может гарантировать получения верного с точки зрения физики процесса результата. Совместное использование физически содержательной модели и методов машинного обучения позволяет избегать проблем подобного рода \cite{kos1}, \cite{kos2}.
Целью настоящей работы является развитие инструментов прокси-моделирования на основе теории фильтрации и элементов машинного обучения. Целевым показателем при разработке месторождений является добыча нефти, которая зависит от двух составляющих: дебита жидкости и доли нефти в добываемой продукции. В рамках настоящего подхода для предсказания дебита жидкости предлагается использовать модель однофазной фильтрации, а для предсказания долей воды и нефти в добываемой продукции – рекуррентные нейронный сети (РНC). Настройка модели для прогнозирования дебита жидкости включает в себя решение задачи восстановления фильтрационных параметров пластовой системы в межскважинном пространстве при помощи сети радиально-базисных функций (РБФ). Задача решалась путем адаптации фильтрационной модели на известные значения пластового давления при заданных расходах жидкости на скважинах. Целью рекуррентной нейронной сети является установление ключевых зависимостей между режимами работы скважин и обводненностью добываемой продукции. Обучение РНC происходило на фактические значения обводненности добываемой жидкости. Для контроля прогностических свойств моделей проведено разбиение моделируемого периода разработки на обучающий и тестовый интервалы.

\section{Mathematical Model}

\subsection{Filtration model}
A 2D incompressible so called "coloured liquid" model of two-phase flow \cite{bas} was used to solve the problem. The use of a "coloured liquid" model assumes: the same viscosity of fluids equal to $\mu$, directly proportional dependence of the relative phase permeability on the saturation of the corresponding phase, the absence of saturation effect on pressure, capillary forces are not taken into account, the pressure of the phases is taken equal. The fluids flow in porous media
equations can be written as follows
\begin{equation} \label{fil}
	\triangledown \cdot \left[\frac{k}{\mu}\triangledown P \right] = \beta^*h\frac{dP}{dt} + \delta_{l}(x,y),
\end{equation}

\begin{equation} \label{bc}
	\delta_{i}(x,y)  = \left\{\begin{array}{crl}
		0, \;\mbox{if}\;(x,y) \notin\ \Gamma_{in}\cup\Gamma_{out},\\
		q_{i\:k}, \;\mbox{if}\;(x,y) \in \Gamma_{in},\\
		q_{i\:aq}, \;\mbox{if}\;(x,y) \in \Gamma_{out},
	\end{array}\right. \quad i = l,w.
\end{equation}
Closing ratios:
\begin{equation} \label{kr}
P = P_0\mbox{,\quad \mbox{if} $t=0$},
\end{equation}
where $k$ is permeability, $P$ is reservoir pressure, $\beta^*$ is
effective compressibility, $h$ is effective thickness, subscript $i$ denotes $l$ -- liquid
(water and oil), $q_k$ is fluid flow rate in $k$th well, $\Gamma_{in}$
is set of coordinates of sources/drain (wells), $\Gamma_{out}$ is
outer boundary, $P_0$ is reservoir pressure at the initial time $t=0$, $q_{aq}$ is specific fluid flow rate through the outer boundary, which can be written as:
\begin{equation*} \label{qaq}
	q_{i\:aq} = \lambda \frac{k}{\mu}(P_{a} - P|_{\Gamma_{out}}),
\end{equation*}
where $P_a$ is on the aquifer ($P_a = P_0$), $\lambda$ is
aquifer productivity factor. It is assumed that the water saturation
of the aquifer is $S_{a\:w} = 1$. The goal of the inverse problem is to find the reservior permeability  $k = k(x,y)$ at which the calculated values of reservoir pressure and water flow rate on the wells will satisfactorily coincide with the initial values.

\subsection{RBF}

Для расчета объема добычи жидкости использована упрощенная по сравнению с исходной постановкой однофазная модель фильтрации. Данная модель связывает информацию об объемах добычи и закачки с информацией об энергетическом состоянии пласта, то есть с замерами пластового давления. 

Адаптация модели на историю происходит за счет настройки фильтрационных параметров пласта. Предлагаемый подход предполагает использование постоянного во времени поля общей подвижности, которое наилучшим образом позволит описать историю разработки. Однако, при вытеснении нефти водой происходит постепенное изменение фазового состава жидкости в пласте, что приводит к изменению общей подвижности. Таким образом, решение обратной задачи сводится к определению некоторого эффективного поля общей подвижности, которое так же (далее) используется для прогнозных расчетов.

Поле общей подвижности  рассматривается в виде функциональной зависимости , где  – вектор пространственных координат. В качестве которой использована сеть радиально базисных функций

\begin{eqnarray}
	h_i(x) = exp \left(\frac{-\lVert x - c_i \rVert^2}{\epsilon^2_i}\right)
\end{eqnarray}

\begin{eqnarray}
	\lambda(x) = w_ih_i + b
\end{eqnarray}
где $x$ – координаты расчетных узлов,  $c_i$– положение базиса, $\epsilon_i$ – ширина базиса, $w_i$ – вес базиса, $b$ – свободный член. Параметры сети радиально базисных функций $c_i$, $\epsilon_i$, $w_i$, $b$  настраиваются в процессе адаптации однофазной фильтрационной модели.

\subsection{INVERSE PROBLEM}

The inverse problem is solved in an optimization
formulation, which consists in minimizing the objective function $J$.
The objective function can be written as a sum of terms, each of which
is the product of a function characterizing the deviation of the
calculated values from the control values and a weighting
coefficient for the corresponding type of variable. The mean square
error (MSE) was chosen as a deviation measure
\begin{equation} \label{mse}
	J=w_p\frac{1}{N}\sum_{i=1}^N{\left(p_c^i-p_f^i\right)^2}+
	w_{q_w}\frac{1}{M}\sum_{i=1}^M{\left(q_{w\:c}^i-q_{w\:f}^i\right)^2},
\end{equation}
where $p_c^i$ is the calculated value of reservoir pressure, $p_f^i$
is the actual value, $i$ is the measurement number, $N$ is the
number of reservoir pressure measurements, $q_{w\:c}^i$ and $
q_{w\:f}^i$ are calculated and actual value of water flow in wells,
respectively, the subscript $f$ means actual values (fact), and $c$
-- calculated (calculated), $M$ is number of flow measurements well
water, $w_p$ and $w_{q_w}$ are weight coefficients taking into
account the degree of influence of various parameters (dimension,
data quality, etc.). The arguments of the term of the objective
function responsible for pressure are the actual (measured) and
calculated values of the reservoir pressure at the well locations at
the given time points. The calculated values, in turn, depend on the
model parameters $u$. The solution is found using the gradient
optimization algorithm and consists in determining the set of model
parameters corresponding to the minimum $J$ and satisfying the
constraints in the form of inequalities
$
u_{min}\leq\ \boldsymbol{u}\leq u_{max},
$
$u_{min}$ and $u_{max}$ are minimum and maximum value for 
parameter.

The optimization problem is solved when each component of the gradient of the objective function tends to 0, which can be written as:
\begin{equation}\label{grad}
	\frac{\partial J}{\partial u_k} = 2w_p\frac{1}{N}\sum_{i=1}^N
	({p_c^i-p_f^i}) \frac{\partial p_c^i}{\partial
		u_k}+2w_{q_w}\frac{1}{M}\sum_{i=1}^M{\left(q_{w\:c}^i-q_{w\:f}^i\right)}\frac{\partial
		q_{w\:c}^i}{\partial u_k}.
\end{equation}
To solve the optimization problem, it is necessary that each
component  of the objective function gradient tends to 0, which can
be written as
\begin{equation} \label{rp}
	\frac{\partial J}{\partial u_k} \rightarrow 0.
\end{equation}
The solution of the inverse problem (\ref{rp}) is found  numerically
by the iterative method. At each iteration, the direct problem
(\ref{fil})--(\ref{bc}) is numerically solved and the derivatives of
the objective function are calculated according to the adjustable
parameters of the \cite{opt} model. The numerical solution was found
by the control volume method for a two-dimensional rectangular
difference grid using the IMPES method \cite{azi}.

To calculate the term of the gradient  component of the objective
function (\ref{grad}), which is responsible for the deviation of the
flow rates of phases (water), it is necessary to convert it to the
form, where the factor $\frac{\partial q_{w\:c}^i}{\partial u_k}$
will be expressed in terms of the derivative formation pressure on
the parameter $u_k$. For a model of "colored liquids", the water flow
rate can be expressed in terms of the liquid flow rate and the
relative phase permeability as
$
q_{w\:c}^i = k_{w}^i \: q_{w\:c}^i.
$
Taking into account Eqs.(\ref{bc}) and (\ref{kr}), the derivative of the
relative phase permeability of water with respect to water
saturation
$
\frac{\partial k_{w}^i}{\partial s_w^i} = 1,
$
where $s_w^i$ is water saturation in the near-well area. Then $\frac{\partial q_{w\:c}^i}{\partial u_k}$ is written as
\begin{equation} \label{dq_du}
	\frac{\partial q_{w\:c}^i}{\partial u_k} = k_{w}^i  \frac{\partial
		q_{l\:c}^i}{\partial u_k} + q_{l\:c}^i \frac{\partial
		k_{w}^i}{\partial u_k} = q_{l\:c}^i \frac{\partial k_{w}^i}{\partial
		s_w^i} \frac{\partial s_w^i}{\partial p_c^i}\frac{\partial
		p_c^i}{\partial u_k} =  q_{l\:c}^i  \frac{\partial s_w^i}{\partial
		p_c^i}\frac{\partial p_c^i}{\partial u_k}.
\end{equation}
In the used model, fluid flow rates are set on the wells, then the term $\frac{\partial q_{w\:c}^i}{\partial u_k}$ in Eq. (\ref{dq_du}) is equal to 0, because the adjustable parameter $u_k$ cannot affect the value of fluid flow rates. In
the case of solving the problem at given bottomhole pressures, this
term must be taken into account.

The main computational complexity consists in calculating the values of
$\frac{\partial p_c^i}{\partial u_k}$ and $\frac{\partial
	s_w^i}{\partial p_c^i}$. Their calculation is performed numerically
using the Eqs. (\ref{fil})--(\ref{fil2}) and based on the \cite{opt}
approach. The used algorithm of derivatives calculation can be use for other methods of reservoir permeability recovery \cite{leg}.

The objective function used to solve the inverse problem
({\ref{mse}}) is not always convenient for a subjective evaluation
of the accuracy of the model. 
%The assessment of the model matching accuracy, as an additional metric, was carried out using the indicator -- the average absolute error in percent "Mean Absolute Percentage Error" (MAPE). 
The evaluation of the history matching and forecasting accuracy was made using the the Mean Absolute Percentage Error (MAPE).
For the values of reservoir pressure and water
flow rate these metrics can be represented as follows
\begin{equation*} \label{mape_p}
	J_{mape(p)}=\frac{1}{N}\sum_{i=1}^N{\left|\frac{p_c^i-p_f^i}{p_f^i}\right|},\quad
	J_{mape(q_w)}=\frac{1}{M}\sum_{i=1}^M{\left|\frac{q_{w\:c}^i-q_{w\:f}^i}{q_{w\:f}^i}\right|}.
\end{equation*}
This metric allows to estimate the accuracy of the model's reproduction of actual reservoir pressure and water flow rate values at wells.



При решении обратной задачи в качестве режимов работы использованы дебиты добывающих скважин и приемистости нагнетательных скважин, полученные из прямого численного моделирования. Пористость, коэффициенты сжимаемости, толщина пласта, начальные и граничные условия соответствовали прямой задаче. Обратная задача решается в оптимизационной постановке, в которой минимизируется целевая функция

\begin{figure}
	\centering
	\includegraphics[width=0.7\linewidth]{eq005}
	\caption{}
	\label{fig:eq005}
\end{figure}

\begin{figure}
	\centering
	\includegraphics[width=0.7\linewidth]{eq006}
	\caption{}
	\label{fig:eq006}
\end{figure}

где  – истинное значение подвижности на скважине,  – расчетное значение подвижности,  – левое и правое ограничение на значение общей подвижности подвижность,  – количество значений подвижности, ,  – расчетное значение пластового давления,  – истинное значение пластового давления,  – весовой коэффициент i-го замера давления,  – количество замеров пластового давления,  и   – весовые коэффициент используются для настройки баланса между составляющими целевой функции.

Первое слагаемое целевой функции отвечает за настройку поля подвижности на известные на скважинах фильтрационные свойства пласта. Причем рассчитанная подвижность может изменяться в диапазоне значений от  до , не приводя к росту целевой функции. Наличие допустимого интервала связано с изменением во времени фазового состава на скважинах, влияющего на подвижность в прискважинной зоне. Второе слагаемое описывает отклонение расчетных давлений от известных значений пластового давления. Весовые коэффициенты  отражают важность i-го замера давления. В настоящем исследовании значения весовых коэффициентов линейно увеличивались со временем, таким образом наиболее важными считались значения давления наиболее близкие к периоду прогноза. При решении обратной задачи в качестве фактических замеров пластового давления использовалось ограниченное количество давлений, полученные из решения прямой задачи. Случайным образом выбраны 10\% от общего количества значений пластового давления, которые выступали в качестве истинных значений.


\section{RESULTS}

Апробация предлагаемых подходов проведена на примере синтетической модели элемента разработки нефтяного пласта. Синтетическая гидродинамическая модель выступала в качестве источника исходных данных, которые были использованы для решения задач адаптации и прогнозирования. Набор данных получен путем прямого численного моделирования двухфазной фильтрации воды и нефти. 

Синтетическая модель представляет собой элемент разработки месторождения размером 1000 на 1000 м, включающий в себя 5 скважин: 3 добывающие (№ 1-3) и 2 нагнетательные (№ 4,5) . На рис. 1 приведена схема расстановки скважин и заданное зонально неоднородное поле проницаемости. Проницаемость в основной части расчетной области составляла 0.1 мкм2. При этом в пласте присутствуют высокопроводящие включения с проницаемостью 1 мкм2, которые соединяют пары добывающая-нагнетательная скважина 1-5 и 2-4. Добывающая скважина 3 изолирована от остальных скважин барьером с проницаемостью 0,05 мкм2.

\begin{figure}
	\centering
	\includegraphics[width=0.7\linewidth]{fig001}
	\caption{Поле абсолютной проницаемости и схема расстановки скважин Fig. 1. Absolute permeability field and well spacing pattern}
	\label{fig:fig001}
\end{figure}

Период моделирования составлял 10 лет, на протяжении которых скважины работали с заданными режимами. Расчетный шаг составлял 1 месяц. Зависимости приемистости нагнетательных скважин от времени, приведенные на рис. 2, получены путем случайной генерации и имеют кусочно-постоянный вид. Так же с заданной периодичностью происходит отключение нагнетательных скважин, период простоя составлял от 2 до 4 месяцев. На добывающих скважинах заданы изменяющиеся по гармоническому закону забойные давления, которые приведены на рис. 3. В начальный момент времени пласт полностью насыщен нефтью, начальное пластовое давление составляло 15 МПа. На границах элемента разработки задано условие непротекания.

\begin{figure}
	\centering
	\includegraphics[width=0.7\linewidth]{fig002}
	\caption{Динамика приемистости нагнетательных скважин 4 и 5 Injectivity dynamics of injection wells 4 and 5}
	\label{fig:fig002}
\end{figure}

\begin{figure}
	\centering
	\includegraphics[width=0.7\linewidth]{fig003}
	\caption{Динамика забойного давления добывающих скважин 1-3 Dynamics of bottomhole pressure of production wells 1-3}
	\label{fig:fig003}
\end{figure}


Параметры горной породы и насыщающих ее флюидов, принятые для расчета, имели следующие значения: пористость 0.15, значения абсолютной проницаемости представлены в виде карты на рис. 1, вязкость воды 1 мПа*c, вязкость нефти 5 мПа*c. Относительные фазовые проницаемости нефти и воды выбраны в виде модельных квадратичных функций . При рассмотрении двухфазной постановки на поле давления оказывает влияние не только фильтрационные свойства пластовой системы, но и сжимаемость насыщающих горную породу флюидов. Основное внимания в настоящем исследовании сконцентрировано на фильтрационных свойствах пластовой системы, поэтому влияние различной сжимаемости фаз исключено из рассмотрения  1/МПа. Сжимаемостью порового пространства горной породы  пренебрегаем. Толщина пласта принята равной 1 м.

Полученный набор данных был использован для решения обратной задачи. При этом для оценки прогностических свойств модели данные были разделены на обучающую и тестовую выборки. Первые 7 лет истории разработки отнесены к обучающей выборке и использованы для адаптации модели. Период 7-10 лет включен в тестовую выборку для которой проведены прогнозные расчеты. Тестовая и обучающая выборки обозначены на рис. 2, 3 и 5 зеленым и красным цветом фона соответственно.


\subsection{Адаптация однофазной фильтрационной модели}




Обратная задача решалась с помощью градиентных методов, которые требуют расчета градиентов целевой функции по настраиваемым параметрам сети радиально базисных функций: . Градиенты вычисляются стандартным для области машинного обучения методом обратного распространения ошибки, который в данном случае предполагает решение сопряженной задач для фильтрационной модели \cite{far}. В качестве инструмента для реализации представленного подхода была использована библиотека для машинного обучения Flux \cite{inn} языка программирования Julia.

Рассмотрим истинное поле общей подвижности, приведенное на рис. 6 в логарифмическом масштабе. Поле соответствует концу обучающей выборки (t = 7 лет). Видно, что поле общей подвижности качественно повторяет поле проницаемости (см. рис. 1). Кроме поля подвижности на рис. 6 представлены связи между скважинами и их значения, которые были получены методом частичного исключения переменных из численной фильтрационной модели [Андреев, 2013]. Наибольшие значения связей 2-4 и 1-5 соответствуют промытым водой высокопроводящим каналам, малые значения связей относятся к изолированной добывающей скважине 3. Коэффициенты связей имеют смысл проводимости между скважинами на единицу толщины пласта. Подобные коэффициенты связи могут быть получены при интерпретации попарного гидропрослушивания скважин.

\begin{figure}
	\centering
	\includegraphics[width=0.7\linewidth]{eq008}
	\caption{Рис. 6. Поле общей подвижности на конце обучающей выборки Fig. 6. Total mobility field at the end of the training set}
	\label{fig:eq008}
\end{figure}


На рис. 7 в логарифмическом масштабе приведено восстановленное поле общей подвижности и соответствующие ему связи между скважинами. Восстановленное поле качественно повторяет истинное поле: между парами скважин 1-5 и 2-4 восстановлены области с высокой подвижностью, скважина 3 находится в области низкой подвижности.

\begin{figure}
	\centering
	\includegraphics[width=0.7\linewidth]{eq009}
	\caption{Восстановленное поле общей подвижности Restored general total field}
	\label{fig:eq009}
\end{figure}


Количественное сопоставление истинного и восстановленного поля общей подвижности проведено на основе рассчитанных для них величин связей между скважинами. Сравнение связей приведено в таблице из которой видно, что ранжирование восстановленных связей полностью совпадает с ранжированием истинных связей. Кроме того, получено удовлетворительное количественное совпадение величин связей: в среднем относительная ошибка составляет 9\%. Таким образом, восстановленное поле общей подвижности качественно и количественно воспроизводит ключевые особенности истинного поля.
\begin{figure}
	\centering
	\includegraphics[width=0.7\linewidth]{eq010}
	\caption{}
	\label{fig:eq010}
\end{figure}


Для реальных пластовых систем величины связей между скважинами, как правило, достоверно не известны. В виду этого одним из основных критерием для оценки качества адаптации модели выступают замеры пластового давления. На рис. 8 приведено сравнение значений фактических и расчетных пластовых давлений, полученных после восстановления поля общей подвижности. Полыми маркерами обозначены значения соответствующие обучающей выборке, закрашенными – тестовой. Видно, что получено удовлетворительное совпадение давлений  на обучающей и тестовой выборке, значения средней относительной ошибки для которых составляет 2.6 и 4.1\% соответственно. Модель демонстрирует удовлетворительную точность на тестовой выборке, что подтверждает ее прогнозные свойства.


\begin{figure}
	\centering
	\includegraphics[width=0.5\linewidth]{eq011}
	\caption{Сопоставление фактических и расчетных значений пластового давления для добывающих и нагнетательных скважин. Полые маркеры соответствуют обучающей выборке, закрашенные – тестовой выборке. Пунктирная линия обозначает интервал отклонения в 10\% Comparison of true and calculated values of reservoir pressure for production and injection wells. Hollow markers correspond to the training set, filled markers correspond to the test set. The dotted line indicates the 10\% deviation interval}
	\label{fig:eq011}
\end{figure}

\section{CONCLUSIONS}

Предложен подход к совместному применению теории фильтрации и машинного обучения для адаптации однофазной модели фильтрации на историю разработки и расчета обводненности добываемой жидкости. На примере синтетической модели элемента нефтяного месторождения с зонально неоднородным полем проницаемости была продемонстрирована реализация предложенных подходов.
Адаптация однофазной модели фильтрации на историю проведена путем восстановления поля общей подвижности с помощью сети радиально-базисных функций. На основе восстановленного поля рассчитаны коэффициенты связи между скважинами, которые качественно и количественно соответствуют истинным связям.
Обводненность добываемой жидкости получена с помощью РНC. Применение рекуррентных нейронных сетей позволило с достаточной точностью спрогнозировать динамику обводнения добывающих скважин. Воспроизведено немонотонное поведение обводненности, которое обусловлено изменением режимов работы скважин, в том числе отключением нагнетательных скважин.
Удовлетворительные прогнозные свойства связки моделей позволяют использовать разработанный подход для оптимизации режимов работы нагнетательных скважин с целью увеличения добычи нефти.

\section{FUNDING}
The research was carried out within the state assignment of Ministry of Science and Higher Education of the Russian Federation (project No. 121030500156-6).

\begin{thebibliography}{20}
	
\bibitem{mus} E.N.Musakaev, S.P.Rodionov, D.Y.Legostaev, V.P.Kosyakov, "Parameter identification for sector filtration model of an oil reservoir with complex structure" // AIP Conference Proceedings 2125,030113 2019;

\bibitem{kos} V. P. Kosyakov. "Structural and Parametric Identification of an Aquifer Model for an Oil Reservoir". Lobachevskii J Math. 2020. V. 41, P. 1242-1247.

\bibitem{bas} K. S. Basniev, N. M. Dmitriev, R. D. Kanevskaya, V. M. Maksimov. \textit{Underground hydromechanics}. M.-Izhevsk: Institute for Computer Research, 2006. [in Russian]

\bibitem{azi} H. Aziz, E. Settari. \textit{Mathematical modeling of reservoir systems}.  M.-Izhevsk: Institute for Computer Research, 2004. [in Russian]

\bibitem{opt} V.P.Kosyakov, S.P.Rodionov "Optimal control of wells on the basis of two-phase filtration equations". Proceedings of MIPT. 2016. V. 8, N 3. P. 79-90.

\bibitem{leg} V. P. Kosyakov,  D. Yu. Legostaev. "Using elements of machine learning to
solve the inverse problem of reconstructing the hydraulic conductivity feld for a fltration
problem". Tyumen State University Herald. Physical and Mathematical Modeling. Oil, Gas,
Energy, V. 8, N 2 (30), P. 129-149.

\bibitem{mus1} Musakaev E. N., Rodionov S. P. Musakaev N. G. 2021. Hierarchical Approach to Identifying Fluid Flow Models in a Heterogeneous Porous Medium // Mathematics. 2021. Vol. 9. No. 24. 3289. https://doi.org/10.3390/math9243289

\bibitem{maz} Mazo A.B., Potashev K. A. 2020. Superelementy. Modelirovanie razrabotki neftyanykh mestorozhdeniy. Moskva: INFRA-M. 219p. [In Russian]

\bibitem{bek} Bekman A. D., Pospelova T. A., Zelenin D. V. 2020. A new approach to water cut forecasting based on results of capacitance resistance modeling // Tyumen State University Herald. Physical and Mathematical Modeling. Oil, Gas, Energy, vol. 6, no. 1 (21), P. 192-207. https://doi.org/10.21684/2411-7978-2020-6-1-192-207 [In Russian]

\bibitem{pot} Potashev K.A., Akhunov R.R., Mazo A.B. 2022. Calculation of the flow rate between wells in the flow model of an oil reservoir using streamlines // Georesources. 24(1), pp. 27–35. https://doi.org/10.18599/grs.2022.1.3 [In Russian]

\bibitem{tem} Temirchev P., Simonov M., Kostoev R., Burnaev E., Oseledets I., Akhmetov A., Margarit A., Sitnikov A., Koroteev D. 2020. Deep neural networks predicting oil movement in a development unit // Journal of Petroleum Science and Engineering. Vol. 184. P. 106513. https://doi.org/10.1016/j.petrol.2019.106513

\bibitem{ill} Illarionov E., Temirchev P., Voloskov D., Kostoev R., Simonov M., Pissarenko D., Orlova D., Koroteev D. 2022. End-to-end neural network approach to 3D reservoir simulation and adaptation // Journal of Petroleum Science and Engineering Vol. 208. P. 109332. https://doi.org/10.1016/j.petrol.2021.109332

\bibitem{uma} Umanovskiy A. W. 2022. Proxy modeling pf reservoir hydrodynamics with graph neural networks // Tyumen State University Herald. Physical and Mathematical Modeling. Oil, Gas, Energy. Vol 8. No 3 (31). P. 155-177. https://doi.org/10.21684/2411-7978-2022-8-3-155-177 [In Russian]

\bibitem{gop} Gopa K., Yamov S., Naugolnov M., Perets D., Simonov M. 2018. Cognitive Analytical System Based on Data-Driven Approach for Mature Reservoir Management // SPE Russian Petroleum Technology Conference (October 15, 2018, Moscow, Russia). https://doi.org/10.2118/191592-18RPTC-MS

\bibitem{kos1} Kosyakov V. P., Legostaev D. Yu., Musakaev E. N. 2021. The problem of the combined use of filtration theory and machine learning elements for solving the inverse problem of restoring the hydraulic conductivity of an oil field // Tyumen State University Herald. Physical and Mathematical Modeling. Oil, Gas, Energy. Vol. 7. No. 2 (26). P. 113-129. https://doi.org/10.21684/2411-7978-2021-7-2-113-129 [In Russian]

\bibitem{kos2} Kosyakov V. P., Legostaev D. Yu. 2022. Using elements of machine learning to solve the inverse problem of reconstructing the hydraulic conductivity feld for a filtration problem // Tyumen State University Herald. Physical and Mathematical Modeling. Oil, Gas, Energy. Vol. 8. No. 2 (30). P. 129-149. https://doi.org/10.21684/2411-7978-2022-8-2-129-149 [In Russian]

\bibitem{far} Farrell P. E., Ham D. A., Funke S. W., Rognes M. E. 2013. Automated Derivation of the Adjoint of High-Level Transient Finite Element Programs // SIAM Journal on Scientific Computing. Vol. 35. No. 4. P. C369-C393. https://doi.org/10.48550/arXiv.1204.5577

\bibitem{inn} Innes M.,  Saba E, Fischer K., Gandhi D., Rudilosso M.C., Joy N.M., Karmali T., Pal A. 2018. Fashionable Modelling with Flux // Viral Shah // ArXiv. https://doi.org/10.48550/arXiv.1811.01457 

\bibitem{and} Andreev V.B. 2013. Numerical methods. М.: MAKS Press. 336 p.

\bibitem{hor} Horchreiter S., Schmidhuber S. 1997. Long short-term memory // Neural computation Neural Computation. Vol. 9. P. 1735-80. https://doi.org/10.1162/neco.1997.9.8.1735

\end{thebibliography}

\end{document}
